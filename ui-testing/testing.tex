\documentclass{article}

\newcommand{\screen}[3]{
    \begin{figure}[h]
    \centering
    \includegraphics[width=#1\textwidth]{#2}
    \caption{#3}
    \end{figure}
}

\usepackage{graphicx}

\title{Real-World UI Testing}
\author{Team Narwhal}
\date{}

\begin{document}
\maketitle

\section{Introduction}
In this document we present a number of real-world tests.
That is, tests that mimic the behaviour that is performed by a typical user in a browser setting.

\section{Sign-in}
\screen{0.8}{sign-in-form}{A screenshot of the sign-in form.}
We expect that on entry of correct login details we are able to successfully login, while incorrect login details results in a message telling us that something is wrong.
As expected, we receive an error message in the top-right corner when entering incorrect login details:
\screen{0.8}{sign-in-fail}{Unsuccessful sign-in attempt.}\\
and the user is logged in successfully if they provide a correct username and password.

\section{View Timetable}
One of the fundamental features of the website is timetables.
We expect that the timetable for a delegate correctly lists their upcoming classes, and allows the user to select certain classes for reminders/cancellation.
More information about a class, and a picture of the room in which the class takes place, is shown in a card to the right of the list of classes.
It is expected that the information presented in this card updates according to what the user has selected.
\screen{1}{timetable}{List of upcoming classes, and a card showing the room and more information.}\\
As expected, the timetable lists all upcoming classes for the delegate.
Upon clicking a class, the card is updated accordingly.

\section{Cancel Module}
As shown in the figure in the above section, the delegate Ben Reed is signed up for a module titled `Class26'.
Suppose that he wishes to cancel his attendance for this module.
To do this, he must click on one instance of a class for this module, go to `Options' on the card, and click the button titled `Cancel Module'.
We expect that after clicking this button, entries for `Class26' are no longer shown in Ben's timetable as he is no longer a member of that module.
The website performed as expected:
\screen{1}{timetable-removed}{Ben Reed's timetable after cancelling the module `Class26'.}

\section{Class Reminder}
Suppose a user wishes to be reminded about an upcoming class via email.
This can be done by selecting a class, going to `Options' and then clicking the button titled `Send Reminder'.
We expect that once the button has been clicked, an email will be sent to the email address linked to the current account, containing a reminder.
This works as expected:
\screen{1}{email}{The email received after requesting a reminder for `Class3'.}

\end{document}
